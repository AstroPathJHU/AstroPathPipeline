\pdfsuppresswarningpagegroup=1

\documentclass{article}

\usepackage{amsmath}
\usepackage{graphicx}
\usepackage[subrefformat=simple,labelformat=simple]{subcaption}
\usepackage{hyperref}
\usepackage{cleveref}
\renewcommand\thesubfigure{(\alph{subfigure})}

\begin{document}
	
\title{Identifying Bad Regions in Microscope Images}
\author{Heshy Roskes}
\date{July 3, 2020}

\section{Introduction}

We would like a way to automatically identify bad regions in images that should be excluded, or at least corrected, before doing any analysis.  For example, a speck of dust on the slide causes light to diffuse over a wide area, or folds in the tissue might make individual cells difficult to identify.

In principle, after the cells have been identified, it might be possible to identify these regions in a general way by looking for clusters of cells that have strange properties that could never arise biologically.  Of course, this has to be done carefully, because the line between ``strange'' properties that indicate bad data and ``interesting'' properties that indicate real biological features is not necessarily easy to draw.  Nevertheless, with this care taken, the general approach is probably worth a try later on.

For now, we try to identify one bad feature at a time.  If we eventually move to a general approach, these studies will help to guide the development of that approach and will also serve as a useful cross check.

\section{Types of bad regions}

\subsection{Dust specks}

Sometimes, a speck of dust on the microscope or slide can cause light to diffuse, resulting in a bright, circular area, as shown in \cref{fig:dustspeck}.  Visually, the dust is easy to distinguish from a cell because it is so much larger.  To detect the dust, we 

\begin{figure}[ht]
	\centering
	\begin{subfigure}{0.45\linewidth}
	\includegraphics[width=\linewidth]{dust}
	\caption{}
	\label{fig:dustspeck}
	\end{subfigure}
	\begin{subfigure}{0.45\linewidth}
	\includegraphics[width=\linewidth]{dustdetected}
	\caption{}
	\label{fig:dustspeckdetected}
	\end{subfigure}
	\caption{\subref{fig:dustspeck} A microscope image with a speck of dust on it. \subref{fig:dustspeckdetected} The same image with the area covered by the dust speck masked in yellow.  Cells in this region should be excluded in data analysis.}
	\label{fig:dust}
\end{figure}

\end{document}
